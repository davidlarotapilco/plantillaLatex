\documentclass[12pt,a4paper]{article}

% incluyendo paquetes
\usepackage[utf8]{inputenc}
\usepackage[spanish]{babel}
\usepackage{libreria}
\usepackage{animate}


\graphicspath{{C:/Users/HUAWEI/Pictures/imagesppt/}} %\incluye todos las imágenes de esa ruta
%\graphicspath{{D:/proyectos_latex/7mo_semestre/gestion_redes/informe_de_redes/main/images}}
\begin{document} % inicio  de documento 



\begin{titlepage}
    %\begin{tikzpicture}[overlay, remember picture]
    %    \fill[red] (10cm,-10cm) rectangle (5cm,-15cm);
    %\end{tikzpicture}
    
    \miRectangulo{-2cm}{-4cm}{2cm}{5cm}{rosado}
%   \miRectangulo{x     }{y }{x1    }{y1   }{color}
    
    \miRectangulo{-1.5cm}{-2cm}{-1.2cm}{23.5cm}{black} % 1
    \miRectangulo{-1.8cm}{23.5cm}{5.7cm}{23.2cm}{black} % 2
    \miRectangulo{5.025cm}{23.5cm}{5.33cm}{20cm}{black} % 3
    \miRectangulo{-2.7cm}{-1.7cm}{4.7cm}{-2cm}{black} % 4
    \miRectangulo{4.2cm}{-2cm}{4.5cm}{10cm}{black} % 5

    \begin{textblock}{100}(100,20)
        \begin{flushright}
        {\huge{\textbf{Universidad Nacional del Altiplano}}}\\
        {\normalsize{\textbf{Educando mentes, Cambiando el mundo}}}
        \end{flushright}
        
    \end{textblock}
    
    \begin{tikzpicture}[remember picture, overlay]
        \node at (current page.north west) [anchor=north west, xshift=120mm, yshift=-47mm] {\includegraphics[width=0.45\textwidth]{\logoright}};
    \end{tikzpicture}
    \begin{textblock}{100}(100,130)
        \begin{flushright}
            {\Large{\textbf{Facultad de Ingeniería Mecánica Eléctrica,
                    Electrónica y Sistemas}}}\\[10pt]
            {\large{\textbf{Escuela Profesional de Ingeniería\\ de Sistemas}}}
        \end{flushright}
    \end{textblock}

    \begin{textblock}{200}(10,163)
        \begin{center}
            
            %\textcolor{azul}{\rule{\linewidth}{0.80mm}}
            % titulo del articulo
            %Monitoreo de la atención de los estudiantes mediante cámaras y celulares dentro del Aula en Puno Perú  \par
            \vspace*{\fill}
                \begin{minipage}{0.9\textwidth}
                    \centering
                    {\Large {\textbf{Portal de música}}}\par
                \end{minipage}
            \vspace*{\fill}
            \textcolor{azul}{\rule{0.5\linewidth}{0.80mm}} \par
            \vspace{8mm}
            {\large{\textbf{ PARALELISMO, CONCURRENCIA Y SISTEMAS DISTRIBUIDOS }}} \\[10pt]
            {\large{\textbf{\textcolor{azul}{Ing. ROMERO FLORES ROBERT ANTONIO }}}} \\[20pt]
            {\large{\textbf{estudiante}}}\\[10pt]
            {\large{\textbf{$\looparrowright$   Larota Pilco David Brahyan  $\looparrowleft$ }}}\\[5pt]
            %{\large{\textbf{$\looparrowright$    Quispe Calcina Royer $\looparrowleft$ }}}\\[5pt]
            %{\large{\textbf{$\looparrowright$    Rojas Alejo Bruno $\looparrowleft$ }}}\\[5pt]
            %{\large{\textbf{$\looparrowright$  $\mathfrak{David\ Brahyan\ Larota\ Pilco}$   $\looparrowleft$ }}}\\[20pt]
            \today

        \end{center}
    \end{textblock}
\end{titlepage}
%//--------------------------------------
%@article{prueba,
%  title={prueba del documento lenguaje Latex},
%  author={Autor},
%  journal={https://www.overleaf.com/},
%  volume={13},
%  number={36},
%  pages={34--36},
%  year={2022}
%\lstset{language=SQL}
%\begin{lstlisting}
%\end{lstlisting}
%\lstset{language=Python}
%\lstinputlisting{hilosBancaria.py}
%}
%
%
%\begin{figure}[h]
%    \centering
%    \includegraphics[width=0.5\textwidth]{images/medicion_con_tacometro.png}
%    \caption{se realizo la medición con el tacómetro} 
%\end{figure}
%
%\begin{tabular}{ l c l }
%Tipo  			& = & 	GL-90L-4B5 \\
%Ip              & = &	55 \\
%Cos  $\varphi$    & = &  	  0.78 \\
%Voltaje         & = &	 230/400V \\
%Potencia	    & = &	2HP \\
%Intensidad    	& = & 	6.1/3.5 \\
%Frecuencia  	& = & 	60HZ \\
%Rpm     		& = &	1680 
%\end{tabular} % incluyendo la caratula
%\tableofcontents % índice automático
\pagestyle{fancy} \mystyle \newpage % Aplicar el estilo de encabezado y pie de página
% inicio del documento  
\newcounter{step}

\begin{figure}[h]
    \centering
    \animategraphics[autoplay,controls,loop,poster=first,palindrome,width=\textwidth]{1}{images/bd}{1}{2}
    \caption{Animación de una secuencia de imágenes con opciones adicionales.}
    \label{fig:animation}
\end{figure}

\begin{comment}
\section{Introducción}
Este informe describe el proceso de creación de un portal musical utilizando tecnologías web modernas. El portal permite a los usuarios buscar y visualizar información sobre diferentes artistas y álbumes. Para ello, hemos empleado XAMPP para la base de datos, y Bootstrap, HTML, PHP y JavaScript para el desarrollo del frontend y backend.
\cite{mysql}
\section{Configuración del Entorno}
\subsection*{XAMPP}
XAMPP es una distribución de Apache fácil de instalar que contiene MariaDB, PHP y Perl. Este entorno nos permite crear una base de datos local para almacenar la información musical.

\subsection*{Pasos para la configuración de XAMPP}
\begin{itemize}
    \item Descargar e instalar XAMPP desde apachefriends.org.
    \item Iniciar el panel de control de XAMPP y arrancar Apache y MySQL.
    \item Acceder a \href{http://localhost/phpmyadmin}{http://localhost/phpmyadmin} para gestionar la base de datos.
   
\end{itemize}

\subsection*{Creación de la Base de Datos}
Paso 
\refstepcounter{step}
\thestep
: En phpMyAdmin, crear una nueva base de datos llamada ``musica''.

Paso 
\refstepcounter{step} \thestep : Crear las tablas necesarias.
\lstset{language=SQL}
\begin{lstlisting}
    CREATE TABLE Usuarios (
    id INT AUTO_INCREMENT PRIMARY KEY,
    nombre VARCHAR(100),
    email VARCHAR(100) UNIQUE,
    contrasena VARCHAR(100),
    fecha_registro DATETIME DEFAULT CURRENT_TIMESTAMP
);

CREATE TABLE Artistas (
    id INT AUTO_INCREMENT PRIMARY KEY,
    nombre VARCHAR(100),
    biografia TEXT
);

CREATE TABLE Albumes (
    id INT AUTO_INCREMENT PRIMARY KEY,
    titulo VARCHAR(100),
    id_artista INT,
    ano INT,
    FOREIGN KEY (id_artista) REFERENCES Artistas(id)
);

CREATE TABLE Generos (
    id INT AUTO_INCREMENT PRIMARY KEY,
    nombre VARCHAR(50)
);

CREATE TABLE Canciones (
    id INT AUTO_INCREMENT PRIMARY KEY,
    titulo VARCHAR(100),
    id_artista INT,
    id_album INT,
    id_genero INT,
    ruta_mp3 VARCHAR(255),
    FOREIGN KEY (id_artista) REFERENCES Artistas(id),
    FOREIGN KEY (id_album) REFERENCES Albumes(id),
    FOREIGN KEY (id_genero) REFERENCES Generos(id)
);

CREATE TABLE ListasReproduccion (
    id INT AUTO_INCREMENT PRIMARY KEY,
    id_usuario INT,
    nombre VARCHAR(100),
    fecha_creacion DATETIME DEFAULT CURRENT_TIMESTAMP,
    FOREIGN KEY (id_usuario) REFERENCES Usuarios(id)
);

CREATE TABLE ListaCanciones (
    id_lista INT,
    id_cancion INT,
    PRIMARY KEY (id_lista, id_cancion),
    FOREIGN KEY (id_lista) REFERENCES ListasReproduccion(id),
    FOREIGN KEY (id_cancion) REFERENCES Canciones(id)
);

\end{lstlisting}

\begin{figure}[!htb]
    \centering
    \caption{Diseño de la base de datos} 
    \includegraphics[width=0.9\textwidth]{images/baseD.png}
\end{figure}

\section{Desarrollo del Frontend}

\subsection*{Bootstrap y HTML}
Bootstrap facilita la creación de una interfaz de usuario atractiva y responsive. Utilizamos HTML para estructurar el contenido.
\cite{Bootstrap}
\subsection*{Estructura básica del HTML:}
\begin{figure}[!htb]
    \centering
    \includegraphics[width=0.8\textwidth]{images/portalPrincipal.png}
    \caption{Pagina html principal} 
\end{figure}

\lstset{language=HTML}
\lstinputlisting{project/index.html}

\section{Desarrollo del Backend}
\subsection*{PHP}
PHP se utiliza para interactuar con la base de datos y manejar la lógica del servidor.
\begin{figure}[!htb]
    \centering
    \includegraphics[width=0.6\textwidth]{images/php.png}
    \caption{evidencia de los archivos de php que manejan el backend} 
\end{figure}
\newpage
\lstset{language=PHP}
\lstinputlisting{project/db_config.php}


\section{Integración de JavaScript para Mostrar Resultados}
\subsection*{JavaScript:}
Utilice JavaScript para manejar dinámicamente los resultados y mostrarlos en la página.

\lstinputlisting{project/js/javascript.js}

El usuario puede ver su lista de reproducciones donde tiene un lista de reproducciones y cada lista de reproducciones tiene lista de canciones y cada lista de canciones tiene canciones
\begin{figure}[!htb]
    \centering
    \includegraphics[width=0.9\textwidth]{images/lista.png}
    \caption{Lista de reproducciones del usuario david} 
\end{figure}


\newpage
\section{Referencias}
\bibliographystyle{apacite}
\bibliography{referencias.bib}

\end{comment}

\end{document}